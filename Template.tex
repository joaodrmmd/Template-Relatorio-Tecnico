%%%%%%%%%%%%%%%%%%%%%%%%%%%%%%%%%%%%%%%%%%%%%%%%%%%%%%%%%%%%%%%%%%%%%%%%%%%%%%%%
% TEMPLATE ABNT - RELATÓRIO TÉCNICO (Segurança da Informação / Pós-Graduação)
% Modelo reutilizável baseado em abnTeX2
%
% VERSÃO CORRIGIDA - Todos os erros resolvidos
%%%%%%%%%%%%%%%%%%%%%%%%%%%%%%%%%%%%%%%%%%%%%%%%%%%%%%%%%%%%%%%%%%%%%%%%%%%%%%%%

\documentclass[
    12pt,
    a4paper,
    oneside,
    english,
    brazil
]{abntex2}

% --- PACOTES ESSENCIAIS ---
\usepackage[utf8]{inputenc}
\usepackage[T1]{fontenc}
\usepackage{lmodern}
\usepackage{graphicx}
\usepackage{float}
\usepackage{microtype}
\usepackage{booktabs}
\usepackage{array}
\usepackage{indentfirst}
\usepackage{minted}
\usepackage{caption}
\usepackage{subcaption}
\usepackage{color}

% --- CONFIGURAÇÃO DE MARGENS (ABNT NBR 14724) ---
\setlrmarginsandblock{3cm}{2cm}{*}
\setulmarginsandblock{3cm}{2cm}{*}
\checkandfixthelayout

% --- CONFIGURAÇÕES DE FORMATAÇÃO ---
\setlength{\parindent}{1.25cm}
\setlength{\parskip}{0.2cm}
\OnehalfSpacing 

% Aponta para a pasta onde as imagens estão
\graphicspath{{./media/}}

% --- DEFINIÇÃO DE COMANDOS PERSONALIZADOS ---
% Define o comando \curso usando makeatletter
\makeatletter
\newcommand{\curso}[1]{\def\curso@content{#1}}
\newcommand{\imprimircurso}{\curso@content}
\makeatother

% --- METADADOS PERSONALIZÁVEIS ---
\titulo{Título do Relatório Técnico-Educacional}
\autor{Nome do Autor}
\local{Recife - PE}
\data{202x}
\instituicao{Universidade Federal de Pernambuco}
\curso{Pós-Graduação Lato Sensu em... }
\tipotrabalho{Relatório Técnico-Educacional}

% --- CONFIGURAÇÃO DE LINKS ---
\hypersetup{
    colorlinks=true,
    linkcolor=black,
    urlcolor=blue,
    pdftitle={Modelo de Relatório Técnico ABNT},
    pdfauthor={Nome do Autor},
}

%%%%%%%%%%%%%%%%%%%%%%%%%%%%%%%%%%%%%%%%%%%%%%%%%%%%%%%%%%%%%%%%%%%%%%%%%%%%%%%%
% COMANDOS E AMBIENTES PERSONALIZADOS
%%%%%%%%%%%%%%%%%%%%%%%%%%%%%%%%%%%%%%%%%%%%%%%%%%%%%%%%%%%%%%%%%%%%%%%%%%%%%%%%

\newenvironment{finding}[1]{
    \begin{quote}
    \noindent\textbf{Achado Técnico — #1}\\
    \itshape
}{
    \end{quote}
}

\newenvironment{faq}{
    \chapter*{FAQ - Perguntas Frequentes}
    \addcontentsline{toc}{chapter}{FAQ - Perguntas Frequentes}
    \begin{description}
}{
    \end{description}
}

%%%%%%%%%%%%%%%%%%%%%%%%%%%%%%%%%%%%%%%%%%%%%%%%%%%%%%%%%%%%%%%%%%%%%%%%%%%%%%%%
% INÍCIO DO DOCUMENTO
%%%%%%%%%%%%%%%%%%%%%%%%%%%%%%%%%%%%%%%%%%%%%%%%%%%%%%%%%%%%%%%%%%%%%%%%%%%%%%%%
\begin{document}

%%%%%%%%%%%%%%%%%%%%%%%%%%%%%%%%%%%%%%%%%%%%%%%%%%%%%%%%%%%%%%%%%%%%%%%%%%%%%%%%
% CAPA PERSONALIZADA
%%%%%%%%%%%%%%%%%%%%%%%%%%%%%%%%%%%%%%%%%%%%%%%%%%%%%%%%%%%%%%%%%%%%%%%%%%%%%%%%
\begin{capa}
    \begin{center}
        % --- LOGO DA INSTITUIÇÃO ---
        \vspace*{-1cm}
        
        % Arquivo: logo_ufpe.eps (ajuste o nome conforme necessário)
        \includegraphics[width=4cm]{logo_ufpe.eps}\\[1cm]

        % --- INSTITUIÇÃO E CURSO ---
        \textbf{\MakeUppercase{\imprimirinstituicao}}\\[0.3cm]
        \MakeUppercase{\imprimircurso}\\[4cm]

        % --- TÍTULO CENTRALIZADO ---
        {\bfseries\Large \imprimirtitulo}\\[4cm]

        % --- AUTOR ---
        \textbf{\imprimirautor}\\[0.2cm]
        \imprimirtipotrabalho\\[4cm]

        % --- LOCAL E DATA ---
        \imprimirlocal\\[0.2cm]
        \imprimirdata
    \end{center}
\end{capa}

%%%%%%%%%%%%%%%%%%%%%%%%%%%%%%%%%%%%%%%%%%%%%%%%%%%%%%%%%%%%%%%%%%%%%%%%%%%%%%%%
% FOLHA DE ROSTO PADRÃO ABNT
%%%%%%%%%%%%%%%%%%%%%%%%%%%%%%%%%%%%%%%%%%%%%%%%%%%%%%%%%%%%%%%%%%%%%%%%%%%%%%%%
\imprimirfolhaderosto*

%%%%%%%%%%%%%%%%%%%%%%%%%%%%%%%%%%%%%%%%%%%%%%%%%%%%%%%%%%%%%%%%%%%%%%%%%%%%%%%%
% SUMÁRIO AUTOMÁTICO
%%%%%%%%%%%%%%%%%%%%%%%%%%%%%%%%%%%%%%%%%%%%%%%%%%%%%%%%%%%%%%%%%%%%%%%%%%%%%%%%
\pdfbookmark[0]{\contentsname}{toc}
\tableofcontents*
\cleardoublepage

%%%%%%%%%%%%%%%%%%%%%%%%%%%%%%%%%%%%%%%%%%%%%%%%%%%%%%%%%%%%%%%%%%%%%%%%%%%%%%%%
% RESUMO
%%%%%%%%%%%%%%%%%%%%%%%%%%%%%%%%%%%%%%%%%%%%%%%%%%%%%%%%%%%%%%%%%%%%%%%%%%%%%%%%
\begin{resumo}
[Insira aqui um resumo conciso do trabalho, descrevendo o objetivo, metodologia, resultados e conclusões principais, em um único parágrafo.]

\textbf{Palavras-chave:} [Insira de 3 a 5 palavras-chave separadas por ponto e vírgula.]
\end{resumo}

%%%%%%%%%%%%%%%%%%%%%%%%%%%%%%%%%%%%%%%%%%%%%%%%%%%%%%%%%%%%%%%%%%%%%%%%%%%%%%%%
% ABSTRACT (opcional)
%%%%%%%%%%%%%%%%%%%%%%%%%%%%%%%%%%%%%%%%%%%%%%%%%%%%%%%%%%%%%%%%%%%%%%%%%%%%%%%%
\begin{resumo}[Abstract]
[Insert here the English version of the abstract, summarizing the same key information as the Portuguese version.]

\textbf{Keywords:} [Insert 3–5 relevant keywords in English.]
\end{resumo}

%%%%%%%%%%%%%%%%%%%%%%%%%%%%%%%%%%%%%%%%%%%%%%%%%%%%%%%%%%%%%%%%%%%%%%%%%%%%%%%%
% CORPO DO TEXTO
%%%%%%%%%%%%%%%%%%%%%%%%%%%%%%%%%%%%%%%%%%%%%%%%%%%%%%%%%%%%%%%%%%%%%%%%%%%%%%%%
\textual

\chapter{Introdução}
[Apresente o contexto geral do tema, a relevância do trabalho, os objetivos e o problema abordado.]

%%%%%%%%%%%%%%%%%%%%%%%%%%%%%%%%%%%%%%%%%%%%%%%%%%%%%%%%%%%%%%%%%%%%%%%%%%%%%%%%
\chapter{Escopo e Objetivos}
[Defina o escopo da análise e os objetivos específicos e gerais do relatório.]

%%%%%%%%%%%%%%%%%%%%%%%%%%%%%%%%%%%%%%%%%%%%%%%%%%%%%%%%%%%%%%%%%%%%%%%%%%%%%%%%
\chapter{Fundamentos Conceituais}
[Apresente os conceitos teóricos, históricos ou técnicos que fundamentam o trabalho. 
Inclua citações, definições e explicações sobre o tema principal.]

%%%%%%%%%%%%%%%%%%%%%%%%%%%%%%%%%%%%%%%%%%%%%%%%%%%%%%%%%%%%%%%%%%%%%%%%%%%%%%%%
\chapter{Metodologia}
[Descreva a metodologia utilizada para conduzir o estudo, experimento ou análise. 
Inclua técnicas, ferramentas, frameworks e critérios de avaliação utilizados.]

%%%%%%%%%%%%%%%%%%%%%%%%%%%%%%%%%%%%%%%%%%%%%%%%%%%%%%%%%%%%%%%%%%%%%%%%%%%%%%%%
\chapter{Análise Técnica ou Estudo de Caso}
[Apresente o estudo prático, experimento ou investigação realizada. 
Inclua figuras, gráficos, tabelas e explicações passo a passo.]

\begin{figure}[H]
    \centering
    % Certifique-se que o arquivo 'exemplo.jpg' está na pasta 'media/'
    % Comente esta linha se não tiver a imagem
    % \includegraphics[width=0.8\textwidth]{exemplo.jpg}
    \caption{[Legenda descritiva da imagem inserida.]}
    \label{fig:exemplo}
\end{figure}

%%%%%%%%%%%%%%%%%%%%%%%%%%%%%%%%%%%%%%%%%%%%%%%%%%%%%%%%%%%%%%%%%%%%%%%%%%%%%%%%
\chapter{Resultados e Discussão}
[Explique os principais resultados obtidos e discuta seu significado, relacionando-os com os objetivos e a teoria apresentada.]

%%%%%%%%%%%%%%%%%%%%%%%%%%%%%%%%%%%%%%%%%%%%%%%%%%%%%%%%%%%%%%%%%%%%%%%%%%%%%%%%
\chapter{Conclusão}
[Apresente as conclusões gerais, destacando os aprendizados e contribuições do trabalho. 
Se aplicável, aponte limitações e possibilidades de estudos futuros.]

%%%%%%%%%%%%%%%%%%%%%%%%%%%%%%%%%%%%%%%%%%%%%%%%%%%%%%%%%%%%%%%%%%%%%%%%%%%%%%%%
% FAQ
%%%%%%%%%%%%%%%%%%%%%%%%%%%%%%%%%%%%%%%%%%%%%%%%%%%%%%%%%%%%%%%%%%%%%%%%%%%%%%%%
\begin{faq}
    \item[O que motivou este relatório?] [Insira uma resposta breve.]
    \item[Quais ferramentas foram utilizadas?] [Liste e descreva resumidamente.]
    \item[Como os resultados podem ser aplicados?] [Descreva aplicações práticas.]
\end{faq}

%%%%%%%%%%%%%%%%%%%%%%%%%%%%%%%%%%%%%%%%%%%%%%%%%%%%%%%%%%%%%%%%%%%%%%%%%%%%%%%%
% APÊNDICES E ANEXOS
%%%%%%%%%%%%%%%%%%%%%%%%%%%%%%%%%%%%%%%%%%%%%%%%%%%%%%%%%%%%%%%%%%%%%%%%%%%%%%%%
\postextual

% Inicia a seção de Apêndices
\begin{apendices}

\chapter{Material Complementar}
[Inclua aqui imagens, tabelas, logs ou scripts complementares usados no trabalho.]

\end{apendices}

% Inicia o ambiente de Anexos
\begin{anexos}

\chapter{Documentos de Referência}
[Anexe documentos ou registros relevantes que sustentem a análise.]

\end{anexos}

%%%%%%%%%%%%%%%%%%%%%%%%%%%%%%%%%%%%%%%%%%%%%%%%%%%%%%%%%%%%%%%%%%%%%%%%%%%%%%%%
% FIM DO DOCUMENTO
%%%%%%%%%%%%%%%%%%%%%%%%%%%%%%%%%%%%%%%%%%%%%%%%%%%%%%%%%%%%%%%%%%%%%%%%%%%%%%%%
\end{document}
